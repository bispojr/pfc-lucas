\section{Introdução}
Este capítulo tem como objetivo apresentar os principais pontos discutidos no trabalho, relacionar os possíveis trabalhos futuros advindos desta pesquisa e avaliar a principal contribuição deste trabalho para a área científica.





\section{Conclusões}

As conclusões devem estabelecer uma descrição sucinta e sintética daquilo que o autor concluiu ao desenvolver sua pesquisa. Deve haver um cuidado para que a mesma não seja óbvia e também que não seja impossível de identificar no texto. Conclusões sobre um ou outro tipo de tecnologia usada, conclusões sobre caminhos que foram tomados na condução da pesquisa são importantes. E, cabe ressaltar que este texto é do autor, portanto não cabe nesta seção a inserção de referências bibliográficas.

\subsection{Quanto à área aplicada}




\subsection{Quanto à área específica}



\section{Trabalhos futuros}


Em todo trabalho científico, vários caminhos podem ser estabelecidos. Porém cabe geralmente ao autor definir um único para viabilizar a produção e divulgação da sua pesquisa. Estes outros caminhos podem ser apresentados nesta seção, detalhando claramente os motivos da não escolha pelos mesmos. 
Também muito importante nesta seção, é vislumbrar o que ainda pode ser realizado na sequência do próprio trabalho. Toda pesquisa é provavelmente infinita, o que a classifica como concluída, é apenas um ponto de parada para sua divulgação. Portanto, outras contribuições são e serão sempre passíveis. É exatamente isso que cria a evolução, o desenvolvimento e gera inovação. Na área da Ciências Exatas, o termo “Estado da Arte” dá a exatidão desta sequencia. 


\section{Considerações finais}

